% Packages
% sudo tlmgr install silence appendixnumberbeamer fira fontaxes mwe noto csquotes babel helvet
%--- Preamble ---------------------------------------------------------%
% Load LaTeX packages
\documentclass[aspectratio=169]{beamer}                    % supports floating text in any location
%\usetheme[noto, showmaxslides, darkmode]{pureminimalistic}
\usetheme[noto, showmaxslides, lightmode]{pureminimalistic}
\graphicspath{{logos/}}

\usepackage[utf8]{inputenc}
\usepackage{csquotes,xpatch}% recommended
%\usepackage[english]{babel}
\usepackage[american]{babel}
\babelprovide[import, main]{english}
\usepackage{tikz}

% \renewcommand{\pageword}{}
% \renewcommand{\logoheader}{\vspace{1.5em}}

\usepackage[
	%    natbib=true,
	backend=biber,
	%    style=abnt,
	%    style=authoryear-comp,
	%    style=authoryear,
	style=ieee,
	%    style=acm,
	%    style=apalike,
	%    style=siam,
	%    style=ieeetr,
	%    style=plain,
	doi=true,
	eprint=false,
	hyperref=true]
{biblatex}
\addbibresource{references.bib}

%% this makes it possible to add backup slides, without counting them
\usepackage{appendixnumberbeamer}
\renewcommand{\appendixname}{\texorpdfstring{\translate{appendix}}{appendix}}
% logos

% footer page
\renewcommand{\pageword}{Slide}

% Math Font Default (Fira is strange)
\renewcommand\mathfamilydefault{\rmdefault}

% if loaded after begin{document} a warning will appear: "pdfauthor already used"
\title[\texttt{OptimalDesign.jl}]{Model-based optimal design of experiments with Pumas: \texttt{OptimalDesign.jl}}
\author[PumasAI]{Jose Storopoli and Mohamed Tarek \quad
	\texttt{[jose.storopoli|mohamed]@pumas.ai}}
\institute{PumasAI}
\date{\today}

\begin{document}
% has to be loaded outside of a frame to work!
\maketitle

% For longer table of contents, I find it cleaner to
% use no footline.
\begin{frame}[plain, noframenumbering]{Outline}
	\tableofcontents
\end{frame}

\section{Motivation}
\begin{frame}{Model-based}
	\begin{vfilleditems}
		\item $M$ -- parametric model.
		\item $\theta \in \Theta$ -- model parameters.
		\item $(M, \Theta)$ -- hypothesis.
	\end{vfilleditems}
\end{frame}

\begin{frame}{Why Optimal Design?}
	\begin{vfilleditems}
		\item Which model $M$ best describes the drug's effect?
		\item Which parameters $\theta$ are the best estimates?
		\item How to quantify the uncertainty in those $\theta$?
	\end{vfilleditems}
\end{frame}

\begin{frame}{Which model $M$ best describes the drug's effect?}
	Not committing to a particular model M and instead collecting data
	that would let us learn the parameter values of multiple models
	$M_1, M_2, \ldots , M_n$ simultaneously.
\end{frame}

\begin{frame}{Which parameters $\theta$ are the best estimates?}
	Not committing to particular parameter values $\theta_i$ for
	each model $M_i$ and instead using a set of values
	$\Theta_i: \theta_i \in \Theta_i$,
	e.g. a discrete set of specific parameter values or a continuous set.
\end{frame}

\begin{frame}{How to quantify the uncertainty in those $\theta$?}
	Using the expected Fisher information matrix (FIM) instead of the
	observed one to estimate the expected standard errors at each parameter
	value $\theta_i \in \Theta_i$.
\end{frame}

\section{Fisher Information Matrix -- FIM}
\begin{frame}{Fisher Information Matrix -- FIM}
	\begin{vfilleditems}
		\item \textbf{Fisher information} is a way of measuring the amount of information that an observable
		random variable $X$ carries about an unknown parameter $\theta$
		upon which the probability of $X$ depends.
		\item Formally, it is the expected value of the observed information,
		which in turn is the \textbf{negative of the second derivative of the loglikelihood}.
		\item if $\theta$ is \textit{not} a scalar, then the information
		is expressed as a matrix, FIM, with the second derivative becoming
		the \textbf{Hessian matrix}:
		$$
			-\operatorname{E}_{p(x \mid \theta)} \left[ \mathbf{H}_{\log p(x \mid \theta)} \right]
		$$
	\end{vfilleditems}
\end{frame}

\begin{frame}{Fisher Information Matrix Properties}
	\begin{vfilleditems}
		\item $N \times N$ positive semidefinite.
		\item symmetric, if second partial derivatives are all continuous.
	\end{vfilleditems}
\end{frame}

\section{Optimal Design Objective}
\begin{frame}{Optimal Design Objective}
	\small
	There are a number of possible objectives that correspond to maximizing the
	information learned in the optimal experiment design:
	\begin{vfilleditems}
		\item \textbf{A-optimality}:
		\begin{vfilleditems}
			\item \textbf{minimizing} the \textbf{trace} of the inverse of the expected FIM.
			\item minimizing the sum of the expected standard errors.
		\end{vfilleditems}
		\item \textbf{D-optimality}:
		\begin{vfilleditems}
			\item \textbf{maximizing} the \textbf{determinant} of the expected FIM.
			\item maximizing the product of the eigenvalues of the expected FIM.
			\item which indirectly minimizes the expected standard errors.
		\end{vfilleditems}
		\item \textbf{T-optimality}:
		\begin{vfilleditems}
			\item \textbf{maximizing} the \textbf{trace} of the expected FIM.
			\item maximizing the sum of the eigenvalues of the expected FIM.
			\item which is also roughly correlated to minimizing the
			sum of the expected standard errors but giving more weight
			to the parameters with smaller standard errors.
		\end{vfilleditems}
	\end{vfilleditems}
\end{frame}

\section{Optimal Design in Pumas}
\begin{frame}{Optimal Design in Pumas}
	\begin{vfilleditems}
		\item There are a number of degrees of freedom that an experiment designer can
		control when designing an experiment.
		\item This leads to different types of optimization problems,
		or so-called optimal design tasks.
		\item Currently\footnote{Pumas version \texttt{2.2}},
		the main optimal design task supported in Pumas is the
		\textbf{sample time optimization}.
	\end{vfilleditems}
\end{frame}

\begin{frame}{Sample Time Optimization}
	The following are assumed to be \textbf{fixed}:
	\begin{vfilleditems}
		\item \textbf{model and parameter values}, e.g. from a similar study.
		\item \textbf{subjects' covariates}, a.k.a. subject templates, e.g.
		typical values in the target population.
		\item \textbf{number of replicas of each subject template},
		using best practices in randomized sampling.
		\item \textbf{dosing regimen of each subject},
		e.g. from initial simulations to avoid toxicity.
		\item \textbf{number of observations per subject template}.
	\end{vfilleditems}
	The \textbf{only degree of freedom allowed to change is which times the observations
		are to be made for each subject template}.
\end{frame}

\begin{frame}{Sample Time Optimization}
	The same optimal design task may be \textbf{repeated for different values of the
		other fixed degrees of freedom},
	e.g. different dosing regimens or number of subjects/observations,
	to find a satisfactory design.
	\vfill
	This parametric study is a form of naive
	\textbf{search-based, bi-level optimization}\footnote{
		Bilevel optimization is a special kind of optimization where one problem is embedded (nested) within another.
		The outer optimization task is commonly referred to as the upper-level optimization task,
		and the inner optimization task is commonly referred to as the lower-level optimization task.}.
\end{frame}

\begin{frame}{Sample Time Optimization -- Constraints}
	Some possible \textbf{constraints} to consider when doing sample time optimization are:
	\begin{vfilleditems}
		\item \textbf{lower and upper bounds on the sample times},
		e.g. the start and end date of the data collection part of the study.
		\item \textbf{minimum offset between any two consecutive observations}.
		\item \textbf{time window constraints},
		e.g. the working hours of the clinical staff.
		\item \textbf{maximum and/or minimum number of measurements per time window}.
	\end{vfilleditems}
\end{frame}
\section{Selected Literature}
\begin{frame}[allowframebreaks]{Selected Literature}
	\printbibliography
\end{frame}

% \appendix % do not count the following slides for the total number
% \section*{Backup Slides}
% \begin{frame}[plain, noframenumbering]
% 	\centering
% 	\vfill
% 	{\fontsize{40}{50}\selectfont Backup Slides}
% 	\vfill
% \end{frame}

\end{document}
